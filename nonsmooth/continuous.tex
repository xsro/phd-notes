\chapter{连续控制系统}


\section{非光滑Lyapunov 方法(Dini导数)}

使用Dini导数表示的非光滑系统李亚普诺夫方法见书\cite{roucheStabilityTheoryLiapunov1977}。
Dini导数可以用于计算连续但是不可导的函数的导数,因此常用于非光滑系统的分析。
Dini导数的定义为:

\begin{definition}[Dini导数]
  \citefrom{\cite[p 659 in proof of theorem 3.4]{khalilNonlinearSystems}}
  The \textbf{upper Dini derivative}, which is also called an upper right-hand derivative, of continuous function  is defined by 
  \begin{equation}
    D^+ v(t) = \limsup_{h\to 0^+} \frac{v(t+h)-v(t)}{h}
  \end{equation}
  where $\limsup_{n\to\infty}$ (the limit superior).
\end{definition}
这里的$\limsup$是上极限,形象地理解是,函数$y=\sin(x)$在无穷处是没有极限的,但是在无穷远处是有上极限$\limsup_{x\to\infty} \sin(x)=1$和下极限$\liminf_{x\to\infty} \sin(x)=-1$的。

\begin{lemma}[非光滑版本的 LaSalle 不变集原理]
  \citefrom{最早出自\cite[Theorem 2]{lasalleStabilityTheoryOrdinary1968}
  也见于\cite[p 243]{roucheStabilityTheoryLiapunov1977}}
  % yuSecondorderConsensusMultiagent2017
  Let $x(t)$ be a solution of $\dot{x} = f(x)$ with $x(0) = x_0 \in \RR^k$, where \tR{$f: U \to \RR^k$ is continuous} with an open subset $U$ of $R^k$, and let $V: U \to R$ be a locally Lipschitz function such that $D^+V(x(t)) ≤ 0$, where $D^+$ denotes the upper Dini derivative. 
  Then, with denoting the positive limit set as $\lambda^+(x_0)$, $\lambda^+(x_0)\cup U$ is contained in the union of all solutions that remain in $S = \{x \in U : D^+V(x(t)) = 0\}$.
\end{lemma}