
对于匀速运动的目标,如果所有个体(包括目标)所组成的网络是连通的,
并且连通的时段足够长,那么可以设计观测器观测目标位置,
观测值将会最终指数收敛到目标位置。

\section{观测器设计}

考虑目标动态为:
\begin{equation}
    \begin{cases}
        \dot{x}_0=v_0\\
        \dot{v}_0=a(t)+a_0(t)+\delta(t), x_0,v_0,\delta\in R^2\\
        y=x_0
    \end{cases}
\end{equation}
其中\(y\)是观测到的输出,\(a(t)\)是加速度输入。
文献\cite{HONG20061177}中规定\(a_0(t)\)是已知的,而\(\delta(t)\)是未知的,
满足\(\norm{\delta(t)}\leq \bar{\delta}\)。

\begin{numcases}{\ }
    \dot{\chi}_i = 
    -k \left[
        \sum_{j\in N_i(t)} a_{ij}(t)(x_i-x_j)+b_i(t)(x_i-x_0)
    \right]+v_i, k>0
    \label{feedback}
    \\
    \dot{v}_i=a_0-\gamma k \left[
        \sum_{j\in N_i(t)} a_{ij}(t)(x_i-x_j)+b_i(t)(x_i-x_0)
    \right]
    \label{v0est}
\end{numcases}
\eqref{feedback}是基于邻居节点的反馈控制律,\eqref{v0est}是目标速度\(v_0\)的估计器。

写成矩阵的形式为:
\begin{numcases}{\ }
\dot{x}=-k(L_\sigma + B_\sigma) \otimes I_2 x
 + k B_\sigma \mathbf{1} \otimes x_0 + v
 \\
 \dot{v}= \mathbf{1} \otimes a_0 - \gamma k (L_\sigma+B_\sigma )\otimes I_2 x
 +\gamma k (B_\sigma \mathbf{1})\otimes x_0
\end{numcases}
这里\(\sigma:[0,\infty)\rightarrow \mathscr{P},\mathscr{P}=\{1,2,\dots,N\}\)为恒值分段函数。
\[
    x=\begin{bmatrix}
        x_1 \\\vdots\\x_n
    \end{bmatrix},
    v=\begin{bmatrix}
        v_1 \\\vdots \\v_n
    \end{bmatrix}
    =I_n\otimes I_2 v
\]
记
\[
    \bar{x}=x-\mathbf{1} \otimes x_0,
    \bar{v}=v-\mathbf{1} \otimes v_0,
    \epsilon=\begin{pmatrix}
        \bar{x}\\\bar{v}
    \end{pmatrix}
\]
于是可以写出偏差动态方程:
\begin{equation}
    \dot{\epsilon} = F_\sigma \epsilon + g
\end{equation}
\[
    F_\sigma=\begin{bmatrix}
        -k(L_\sigma+B_\sigma) & I_n\\
        -\gamma k (L_\sigma + B_\sigma) & 0
    \end{bmatrix}\otimes I_2,
    g=\begin{bmatrix}
        0 \\ - \mathbf{1} \otimes \delta
    \end{bmatrix}
\]

\section{匀速运动目标指数收敛}

文献\cite{Kou2022tac,Kou2021}中使用的目标为匀速目标,因此可以忽略\(g\),在此情况下估计位置值指数收敛到目标实际位置。

记\(M_\sigma=L_\sigma+B_\sigma\),
\begin{equation}
    \dot{\epsilon}=F_\sigma \epsilon,F_\sigma=\begin{bmatrix}
        -kM_\sigma & I_n\\
        -\gamma k M_\sigma & 0
    \end{bmatrix}\otimes I_2
\end{equation}


选取如下Lyapunov函数,并做求导运算:
\begin{equation}
    V(\epsilon)=\epsilon^T P \epsilon,
    P=\begin{bmatrix}
        \gamma_1 I_n & - \gamma I_n\\
        -\gamma I_n & I_n\\
    \end{bmatrix},
    \gamma_1>\gamma^2
\end{equation}
\begin{gather}
    \dot{V}(\epsilon)=
    (F_\sigma \epsilon)^TP\epsilon
    +\epsilon^T P (F_\sigma \epsilon)=-\epsilon^T Q \epsilon,\\
    Q_\sigma=\begin{bmatrix}
        2k(\gamma_1-\gamma^2)M_\sigma & -\gamma_1 I_n \\
        -\gamma_1 I_n & 2 \gamma I_N
    \end{bmatrix}
    \otimes I_2
\end{gather}

\begin{lemma}
    \cite{HONG20061177}
    对于连通的网络,矩阵\(M_\sigma\)正定,其最小的特征值大于零,\(\bar{\lambda}=\lambda_{min}(M_\sigma)>0\)
\end{lemma}

选择\(k>\frac{\gamma_1^2}{4\gamma(\gamma_1-\gamma^2)\bar{\lambda}}\),
于是有\[
    Q_\sigma/(2k(\gamma_1-\gamma^2)M_\sigma)=2\gamma I_n - \frac{\gamma_1^2}{2k(\gamma_1-\gamma^2)}M_\sigma^{-1} >0
\]

\begin{theorem}
    Schur分解\footnote{
    Schur complement - Wikipedia. (2022) Retrieved December 12, 2022, from \url{https://en.wikipedia.org/wiki/Schur_complement}
    }
    对于矩阵\(X=\begin{bmatrix}
        A & B \\ B^T & C
    \end{bmatrix}\),有:
    当\(A\)可逆时,当且仅当A和X/A正定时,矩阵X正定:
    \begin{equation}
        X>0 \leftrightarrow 
        A>0,
        X/A=C-B^TA^{-1}B>0
    \end{equation}
\end{theorem}
因此\(Q_\sigma\)正定。

于是:
\begin{gather}
    V(\epsilon)=\epsilon^T P \epsilon \leq \epsilon^T \lambda_{max} (P) \epsilon \\
    \dot{V}(\epsilon)=-\epsilon^T Q \epsilon \leq -\epsilon^T \lambda_{min}(Q) \epsilon
    \leq - \epsilon^T(\beta \lambda_{max}(P))\epsilon
    \\
    \dot{V}(\epsilon) \leq -\beta V(\epsilon) \ for\ 0<\beta\leq \lambda_{min}(Q)/\lambda_{max} (P)
\end{gather}

因此,使用解微分方程的方法,分离变量再积分,得:
\[
    \dot{V}(\epsilon(t))\leq V(x(0))e^{-\beta t}
\]
\[
    \begin{aligned}
        \norm{\epsilon(t)}
    &\leq \left[\frac{V(\epsilon(t))}{\lambda_{min}(P)} \right]^\frac{1}{2}
    \\&\leq \left[\frac{V(x(0))e^{-\beta t}}{\lambda_{min}(P)} \right]^\frac{1}{2}
    \\&\leq \left[\frac{\lambda_{max}(P)\norm{\epsilon(0)}^2e^{-\beta t}}{\lambda_{min}(P)} \right]^\frac{1}{2}
    \\&=\left(\frac{\lambda_{max}(P)}{\lambda_{min}(P)}\right)^\frac{1}{2}
    e^{-\frac{\beta}{2}t}
    \norm{\epsilon(0)}
\end{aligned}
\]

\[\lim_{t\to\infty} V(y(t))=0,\ \lim_{t\to\infty} \chi_i=x_0, exponentially\]

\section{下一步工作}

\begin{itemize}
    \item 目标估计器器指数收敛,控制器指数收敛到估计值,那么总体算法也是指数收敛;
    \item 是不是可以考虑文献\cite{HONG20061177}的目标动态,做一个一般收敛的结果。
\end{itemize}

\printbibliography[heading=subbibintoc]