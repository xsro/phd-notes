\chapter{预备知识}  

\section{控制对象}

\subsection{二阶无人小车模型}\label{subsec:car2}

参考文献\cite{renDistributedConsensusMultivehicle2008}(198页)中的移动机器人运动学模型以及
文献\cite{chenMultiMobileRobot2020}中的动力学模型,
记机器人重心为$\bm{p}_i=[x_i,y_i]^T$,机器人几何中心为
\begin{equation}
    \bm{r}_i=\bm{p}_i+\delta \begin{bmatrix}
        \cos\theta_i \\\sin\theta_i
    \end{bmatrix},
    \label{car:offset}
\end{equation}
其中$\theta_i$为机器人$i$的朝向角,$\delta$为机器人重心与几何中心的距离。


合围机器人的运动学模型为\(\dot{p}_i=v_i[\cos \theta_i,\sin \theta_i]^T\),
合围机器人的动力学模型为\(f_i=m_i \dot{v}_i,\tau_i=J_i \ddot{\theta}_i \)。
于是合围机器人的数学模型可以表示为
\begin{equation}
    \begin{cases}
        \dot{x}_i=v_i \cos \theta_i , \quad \dot{y}_i=v_i \cos \theta_i \\
        \dot{v}_i=f_i/m_i,~\dot{\theta}_i=\omega_i,~\dot\omega_i=\tau_i/J_i
    ,
    \end{cases}
    \label{car:model}
\end{equation}
其中$f_i,\tau_i$分别为小车受到的合外力和合外力矩。
由于移动机器人受到非完整的运动学约束,合外力的方向必须与小车朝向一致。

对\eqref{car:offset}求导,并带入\eqref{car:model}可得

\begin{equation}
    \begin{aligned}
        \dot{\bm{r}}_i
         & =v_i \begin{bmatrix}\cos\theta_i \\ \sin\theta_i\end{bmatrix}
        +\delta \begin{bmatrix}-\sin\theta_i \\ \cos\theta_i\end{bmatrix}\dot{\theta}_i
        \\
        \ddot{\bm{r}}_i
         & =\dot{v}_i \begin{bmatrix} \cos\theta_i \\ \sin\theta_i\end{bmatrix}
        +v_i \begin{bmatrix} -\sin\theta_i \\ \cos\theta_i\end{bmatrix} \omega_i
        +\delta \begin{bmatrix}-\sin\theta_i \\ \cos\theta_i\end{bmatrix}\ddot{\theta}_i
        +\delta \begin{bmatrix}-\cos\theta_i \\ -\sin\theta_i\end{bmatrix}\dot{\theta}_i
        \\
         & =\begin{bmatrix}
                \cos\theta_i & -\sin\theta_i \\
                \sin\theta_i & \cos\theta_i
            \end{bmatrix}
        \begin{bmatrix}
            1/m_i &            \\
                  & \delta/J_i
        \end{bmatrix}
        \begin{bmatrix}
            f_i \\
            \tau_i
        \end{bmatrix}
        + \begin{bmatrix}
              -v_i \omega_i \sin \theta_i - \delta \omega_i^2 \sin \theta_i \\
              v_i \omega_i \cos\theta_i - \delta \omega_i^2 \cos \theta_i
          \end{bmatrix}
        .
    \end{aligned}
    \label{car:linearization1}
\end{equation}
记
\begin{equation}
    R(\theta_i)=\begin{bmatrix}
        \cos\theta_i & -\sin\theta_i \\
        \sin\theta_i & \cos\theta_i
    \end{bmatrix},
    Q=  \begin{bmatrix}
        1/m_i &            \\
              & \delta/J_i
    \end{bmatrix},
    P(v_i,\omega_i,\omega_i)=\begin{bmatrix}
        -v_i \omega_i \sin \theta_i - \delta \omega_i^2 \sin \theta_i \\
        v_i \omega_i \cos\theta_i - \delta \omega_i^2 \cos \theta_i
    \end{bmatrix}.
\end{equation}
同时,选择控制输入为$u_i=[f_i,\tau_i]^T$,
于是\eqref{car:linearization1}可以写作
\begin{equation}
    \ddot{\bm{r}}_i=R(\theta_i) Q \bm{u} + P(v_i,\omega_i,\omega_i)
    .
\end{equation}


利用线性化的结论,可以对控制输入进行变换,
\begin{equation}
    \bm{u}_i=Q^{-1} \bm{r}^{-1}(\theta_i)[
            \bm{u}_i^0
            -P(v_i,\omega_i,\omega_i)
        ].
    \label{car:linearizationInput}
\end{equation}
于是合围机器人的模型可以写为
\begin{equation}
    \begin{cases}
        \dot{\bm{r}}_i=\bm{v}_i \\
        \dot{\bm{v}}_i=\bm{u}^0_i .
    \end{cases}
    \label{eq: double integrator}
\end{equation}
  
\begin{refsection}  
\printbibliography[heading=subbibliography]  
\end{refsection}  